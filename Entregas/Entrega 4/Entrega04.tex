\documentclass[fleqn]{article}

%\pgfplotsset{compat=1.17}

\usepackage{mathexam}
\usepackage{amsmath}
\usepackage{amsfonts}
\usepackage{graphicx}
\usepackage{systeme}
\usepackage{microtype}
\usepackage{multirow}
\usepackage{pgfplots}
\usepackage{listings}
\usepackage{tikz}
\usepackage{dsfont} %Numeros reales, naturales...
\usepackage{cancel}
\usepackage{babel}

%\graphicspath{{images, }}
\newcommand*{\QED}{\hfill\ensuremath{\square}}

%Estructura de ecuaciones
\setlength{\textwidth}{15cm} \setlength{\oddsidemargin}{5mm}
\setlength{\textheight}{23cm} \setlength{\topmargin}{-1cm}



\author{David García Curbelo}
\title{Criptografía}

\pagestyle{empty}


\def\R{\mathds{R}}
\def\Z{\mathds{Z}}
\def\N{\mathds{N}}
\def\Q{\mathds{Q}}
\def\F{\mathds{F}}

\def\sup{$^2$}

\def\next{\quad \Rightarrow \quad}


\begin{document}
    \begin{center}
        \LARGE{\textbf{Ejercicio 3}} \\
        \Large{David García Curbelo} 
    \end{center}

    \vspace{1cm}     

    \textit{Sea $\F_{32} = \F_2 [\xi]_{\xi^5 + \xi^2 + 1}$. Cada uno de vosotros, de acuerdo a su número de $DNI = 45352581$ o similar,
            dispone de una curva elíptica sobre $\F_{32}$ con una raíz $x$ y un punto base dados en el Cuadro 6.1.}

    \vspace{0.5cm}   

    \textbf{Ejercicio 1. }\textit{Calcula, mediante el algoritmo de Shank o mediante el Algoritmo 9, $\log_Q \mathcal{O}$.}

    \vspace{0.5cm}

    \textbf{Ejercicio 2. }\textit{Para tu curva y tu punto base, genera un par de claves pública/privada para el protocolo ECDH.}

    \vspace{0.5cm}

    \textbf{Ejercicio 3. }\textit{Cifra el mensaje $(\xi^3 + \xi^2 + 1, \xi^4 + \xi^2)$ mediante el criptosistema de Menezes-Vanstone.}

    \vspace{0.5cm}

    \textbf{Ejercicio 4. }\textit{Descifra el mensaje anterior.}



\end{document}