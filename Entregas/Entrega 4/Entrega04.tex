\documentclass[fleqn]{article}

%\pgfplotsset{compat=1.17}

\usepackage{mathexam}
\usepackage{amsmath}
\usepackage{amsfonts}
\usepackage{graphicx}
\usepackage{systeme}
\usepackage{microtype}
\usepackage{multirow}
\usepackage{pgfplots}
\usepackage{listings}
\usepackage{tikz}
\usepackage{dsfont} %Numeros reales, naturales...
\usepackage{cancel}
\usepackage{babel}

%\graphicspath{{images, }}
\newcommand*{\QED}{\hfill\ensuremath{\square}}

%Estructura de ecuaciones
\setlength{\textwidth}{15cm} \setlength{\oddsidemargin}{5mm}
\setlength{\textheight}{23cm} \setlength{\topmargin}{-1cm}



\author{David García Curbelo}
\title{Criptografía}

\pagestyle{empty}


\def\R{\mathds{R}}
\def\Z{\mathds{Z}}
\def\N{\mathds{N}}
\def\Q{\mathds{Q}}
\def\F{\mathds{F}}

\def\sup{$^2$}

\def\next{\quad \Rightarrow \quad}


\begin{document}
    \begin{center}
        \LARGE{\textbf{Ejercicio 3}} \\
        \Large{David García Curbelo} 
    \end{center}

    \vspace{1cm}     

    \textit{Sea $\F_{32} = \F_2 [\xi]_{\xi^5 + \xi^2 + 1}$. Cada uno de vosotros, de acuerdo a su número de $DNI = 45352581$ o similar,
            dispone de una curva elíptica sobre $\F_{32}$ con una raíz $x$ y un punto base dados en el Cuadro 6.1.}

    \vspace{0.5cm}   

    \textbf{Ejercicio 1. }\textit{Calcula, mediante el algoritmo de Shank o mediante el Algoritmo 9, $\log_Q \mathcal{O}$.}
    \vspace{0.5cm}

    Teniendo el $DNI = 45352581$, tenemos que $DNI \equiv 5 \pmod{32}$, y por tanto, de acuerdo con el Cuadro 6.1, obtenemos 
    $E = E(\xi^2 + 1, \xi^4 + \xi^3 + \xi + 1)$ y  el punto $Q = (\xi^3 + \xi^2 + \xi, \xi + 1)$. Procedemos al cáclulo del logaritmo 
    $\log_Q \mathcal{O}$ mediante el algoritmo de Shank, por lo que para ello procedemos primeramente al cálculo de las potencias de
    $\xi$ en base $\xi^5 + \xi^2 + 1$.

    \begin{center}
        \begin{tabular}{r l}
            $\xi^0$    = & $1$ \\
            $\xi^1$    = & $\xi$ \\
            $\xi^2$    = & $\xi^2$ \\
            $\xi^3$    = & $\xi^3$ \\
            $\xi^4$    = & $\xi^4$ \\
            $\xi^5$    = & $\xi^2 + 1$ \\
            $\xi^6$    = & $\xi^3 + \xi$ \\
            $\xi^7$    = & $\xi^4 + \xi^2$ \\
            $\xi^8$    = & $\xi^3 + \xi^2 + 1$ \\
            $\xi^9$    = & $\xi^4 + \xi^3 + \xi$ \\
            $\xi^{10}$ = & $\xi^4 + 1$ \\
            $\xi^{11}$ = & $\xi^2 + \xi + 1$ \\
            $\xi^{12}$ = & $\xi^3 + \xi^2 + \xi$ \\
            $\xi^{13}$ = & $\xi^4 + \xi^3 + \xi^2$ \\
            $\xi^{14}$ = & $\xi^4 + \xi^3 + \xi^2 + 1$ \\
            $\xi^{15}$ = & $\xi^4 + \xi^3 + \xi^2 + \xi + 1$ \\
            $\xi^{16}$ = & $\xi^4 + \xi^3 + \xi + 1$ \\
            $\xi^{17}$ = & $\xi^4 + \xi + 1$ \\
            $\xi^{18}$ = & $\xi + 1$ \\
            $\xi^{19}$ = & $\xi^2 + \xi$ \\
            $\xi^{20}$ = & $\xi^3 + \xi^2$ \\
            $\xi^{21}$ = & $\xi^4 + \xi^3$ \\
            $\xi^{22}$ = & $\xi^4 + \xi^2 + 1$ \\
            $\xi^{23}$ = & $\xi^3 + \xi^2 + \xi + 1$ \\
            $\xi^{24}$ = & $\xi^4 + \xi^3 + \xi^2 + \xi$ \\
            $\xi^{25}$ = & $\xi^4 + \xi^3 + 1$ \\
            $\xi^{26}$ = & $\xi^4 + \xi^2 + \xi + 1$ \\
            $\xi^{27}$ = & $\xi^3 + \xi + 1$ \\
            $\xi^{28}$ = & $\xi^4 + \xi^2 + \xi$ \\
            $\xi^{29}$ = & $\xi^3 + 1$ \\
            $\xi^{30}$ = & $\xi^4 + \xi$ \\ \hline
        \end{tabular}
    \end{center}

    Tenemos por tanto que $E = E(\xi^2 + 1, \xi^4 + \xi^3 + \xi + 1) = E(\xi^5, \xi^{16})$ y  el punto $Q = (\xi^3 + \xi^2 + \xi, \xi + 1) = 
    (\xi^9, \xi^{18})$. Ahora procedemos a buscar una cota para $|E| \leq q + 1 + \lfloor 2 \sqrt{q} \rfloor = 32 + 1 + 11 = 44$ (donde $q = 32$).
    Obtenemos así que $f = \lceil 44 \rceil = 7$, por lo que obtenemos los siguientes puntos:
    \begin{center}
        \begin{tabular}{| c | c |}
            \hline
            0 & 0 \\
            1 & $Q$ \\
            2 & $2Q$ \\
            3 & $3Q$ \\
            4 & $4Q$ \\
            5 & $5Q$ \\
            6 & $6Q$ \\ \hline
        \end{tabular}
    \end{center}

    Procedmemos a su cálculo explícito:

    \begin{itemize}
        \item[$2Q$] $ = Q + Q = (\xi^9, \xi^{18}) + (\xi^9, \xi^{18})$
                    \begin{itemize}
                        \item[$\lambda$] $ = x_1 + y_1 x_1^{-1} = \xi^9 + \xi^{18} \cdot \xi^{-9} = \xi^9 + \xi^9 = \xi^4 + \xi^3 + \xi + \xi^4 + \xi^3 + \xi = 0$
                        \item[$x_3$] $ = \lambda^2 + \lambda + a + x_1 + x_2 = \xi^5 + \xi^9 + \xi^{18} = (\xi^2 + 1) + (\xi^4 + \xi^3 + \xi) + (\xi + 1) = \xi^4 + \xi^3 + \xi^2 = \xi^{13}$
                        \item[$y_3$] $ = \lambda(x_1 + x_3 )+ x_3 + y_1 = $
                    \end{itemize}
    \end{itemize}

    \newpage
    \textbf{Ejercicio 2. }\textit{Para tu curva y tu punto base, genera un par de claves pública/privada para el protocolo ECDH.}
    \vspace{0.5cm}


    \newpage
    \textbf{Ejercicio 3. }\textit{Cifra el mensaje $(\xi^3 + \xi^2 + 1, \xi^4 + \xi^2)$ mediante el criptosistema de Menezes-Vanstone.}
    \vspace{0.5cm}


    \newpage
    \textbf{Ejercicio 4. }\textit{Descifra el mensaje anterior.}



\end{document}