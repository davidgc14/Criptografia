\documentclass[fleqn]{article}

%\pgfplotsset{compat=1.17}

\usepackage{mathexam}
\usepackage{amsmath}
\usepackage{amsfonts}
\usepackage{graphicx}
\usepackage{systeme}
\usepackage{microtype}
\usepackage{multirow}
\usepackage{pgfplots}
\usepackage{listings}
\usepackage{tikz}
\usepackage{dsfont} %Numeros reales, naturales...
\usepackage{cancel}
\usepackage{babel}

%\graphicspath{{images, }}
\newcommand*{\QED}{\hfill\ensuremath{\square}}

%Estructura de ecuaciones
\setlength{\textwidth}{15cm} \setlength{\oddsidemargin}{5mm}
\setlength{\textheight}{23cm} \setlength{\topmargin}{-1cm}



\author{David García Curbelo}
\title{Criptografía}

\pagestyle{empty}


\def\R{\mathds{R}}
\def\Z{\mathds{Z}}
\def\N{\mathds{N}}
\def\Q{\mathds{Q}}
\def\F{\mathds{F}}

\def\sup{$^2$}

\def\next{\quad \Rightarrow \quad}


\begin{document}
    \begin{center}
        \LARGE{\textbf{Ejercicio 3}} \\
        \Large{David García Curbelo} 
    \end{center}

    \vspace{1cm}     

    \textit{Los parámetros de un criptosistema de ElGamal son $p = 211$ y 
    $g = 3$, es decir, el criptosistema está diseñado en el cuerpo $\F_{211} = \Z_{211}$ y tomamos
    como generador de $\F_{211}^*$, $g = 3$. La clave pública empleada es $3^a = 109 \pmod{211}$.
    Descifra el criptograma $(154, \text{dni} \pmod{211})$, donde dni es el número de tu DNI. Para 
    calcular los logaritmos discretos necesarios emplea dos de los métodos descritos en la teoría.} 

    \vspace{0.5cm}   



\end{document}