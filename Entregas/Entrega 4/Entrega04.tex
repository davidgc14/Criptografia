\documentclass[fleqn]{article}

%\pgfplotsset{compat=1.17}

\usepackage{mathexam}
\usepackage{amsmath}
\usepackage{amsfonts}
\usepackage{graphicx}
\usepackage{systeme}
\usepackage{microtype}
\usepackage{multirow}
\usepackage{pgfplots}
\usepackage{listings}
\usepackage{tikz}
\usepackage{dsfont} %Numeros reales, naturales...
\usepackage{cancel}
\usepackage{babel}

%\graphicspath{{images, }}
\newcommand*{\QED}{\hfill\ensuremath{\square}}

%Estructura de ecuaciones
\setlength{\textwidth}{15cm} \setlength{\oddsidemargin}{5mm}
\setlength{\textheight}{23cm} \setlength{\topmargin}{-1cm}



\author{David García Curbelo}
\title{Criptografía}

\pagestyle{empty}


\def\R{\mathds{R}}
\def\Z{\mathds{Z}}
\def\N{\mathds{N}}
\def\Q{\mathds{Q}}
\def\F{\mathds{F}}

\def\sup{$^2$}

\def\next{\quad \Rightarrow \quad}


\begin{document}
    \begin{center}
        \LARGE{\textbf{Ejercicio 4}} \\
        \Large{David García Curbelo} 
    \end{center}

    \vspace{1cm}     

    \textit{Sea $\F_{32} = \F_2 [\xi]_{\xi^5 + \xi^2 + 1}$. Cada uno de vosotros, de acuerdo a su número de $DNI = 45352581$ o similar,
            dispone de una curva elíptica sobre $\F_{32}$ con una raíz $x$ y un punto base dados en el Cuadro 6.1.}

    \vspace{0.5cm}   

    \textbf{Ejercicio 1. }\textit{Calcula, mediante el algoritmo de Shank o mediante el Algoritmo 9, $\log_Q \mathcal{O}$.}
    \vspace{0.5cm}

    Teniendo el $DNI = 45352581$, tenemos que $DNI \equiv 5 \pmod{32}$, y por tanto, de acuerdo con el Cuadro 6.1, obtenemos 
    $E = E(\xi^2 + 1, \xi^4 + \xi^3 + \xi + 1)$ y  el punto $Q = (\xi^3 + \xi^2 + \xi, \xi + 1)$. Procedemos al cáclulo del logaritmo 
    $\log_Q \mathcal{O}$ mediante el algoritmo de Shank, por lo que para ello procedemos primeramente al cálculo de las potencias de
    $\xi$ en base $\xi^5 + \xi^2 + 1$.

    \begin{center}
        \begin{tabular}{r l}
            $\xi^0$    = & $1$ \\
            $\xi^1$    = & $\xi$ \\
            $\xi^2$    = & $\xi^2$ \\
            $\xi^3$    = & $\xi^3$ \\
            $\xi^4$    = & $\xi^4$ \\
            $\xi^5$    = & $\xi^2 + 1$ \\
            $\xi^6$    = & $\xi^3 + \xi$ \\
            $\xi^7$    = & $\xi^4 + \xi^2$ \\
            $\xi^8$    = & $\xi^3 + \xi^2 + 1$ \\
            $\xi^9$    = & $\xi^4 + \xi^3 + \xi$ \\
            $\xi^{10}$ = & $\xi^4 + 1$ \\
            $\xi^{11}$ = & $\xi^2 + \xi + 1$ \\
            $\xi^{12}$ = & $\xi^3 + \xi^2 + \xi$ \\
            $\xi^{13}$ = & $\xi^4 + \xi^3 + \xi^2$ \\
            $\xi^{14}$ = & $\xi^4 + \xi^3 + \xi^2 + 1$ \\
            $\xi^{15}$ = & $\xi^4 + \xi^3 + \xi^2 + \xi + 1$ \\
            $\xi^{16}$ = & $\xi^4 + \xi^3 + \xi + 1$ \\
            $\xi^{17}$ = & $\xi^4 + \xi + 1$ \\
            $\xi^{18}$ = & $\xi + 1$ \\
            $\xi^{19}$ = & $\xi^2 + \xi$ \\
            $\xi^{20}$ = & $\xi^3 + \xi^2$ \\
            $\xi^{21}$ = & $\xi^4 + \xi^3$ \\
            $\xi^{22}$ = & $\xi^4 + \xi^2 + 1$ \\
            $\xi^{23}$ = & $\xi^3 + \xi^2 + \xi + 1$ \\
            $\xi^{24}$ = & $\xi^4 + \xi^3 + \xi^2 + \xi$ \\
            $\xi^{25}$ = & $\xi^4 + \xi^3 + 1$ \\
            $\xi^{26}$ = & $\xi^4 + \xi^2 + \xi + 1$ \\
            $\xi^{27}$ = & $\xi^3 + \xi + 1$ \\
            $\xi^{28}$ = & $\xi^4 + \xi^2 + \xi$ \\
            $\xi^{29}$ = & $\xi^3 + 1$ \\
            $\xi^{30}$ = & $\xi^4 + \xi$ \\ \hline
        \end{tabular}
    \end{center}

    Tenemos por tanto que $E = E(\xi^2 + 1, \xi^4 + \xi^3 + \xi + 1) = E(\xi^5, \xi^{16})$ y  el punto $Q = (\xi^3 + \xi^2 + \xi, \xi + 1) = 
    (\xi^9, \xi^{18})$. Ahora procedemos a buscar una cota para $|E| \leq q + 1 + \lfloor 2 \sqrt{q} \rfloor = 32 + 1 + 11 = 44$ (donde $q = 32$).
    Obtenemos así que $f = \lceil 44 \rceil = 7$, por lo que obtenemos los siguientes puntos:
    \begin{center}
        \begin{tabular}{| c | c |}
            \hline
            0 & 0 \\
            1 & $Q$ \\
            2 & $2Q$ \\
            3 & $3Q$ \\
            4 & $4Q$ \\
            5 & $5Q$ \\
            6 & $6Q$ \\ \hline
        \end{tabular}
    \end{center}

    Procedmemos a su cálculo explícito:

    \begin{itemize}
        \item[$2Q$] $ = Q + Q = (\xi^9, \xi^{18}) + (\xi^9, \xi^{18})$
                    \begin{itemize}
                        \item[$\lambda$] $ = x_1 + y_1 x_1^{-1} = \xi^9 + \xi^{18} \xi^{-9} = \xi^9 + \xi^9 = 0$
                        \item[$x_3$] $ = \lambda^2 + \lambda + a + x_1 + x_2 = \xi^5 + \xi^9 + \xi^{9} = \xi^{5}$
                        \item[$y_3$] $ = \lambda(x_1 + x_3) + x_3 + y_1 = \xi^{5} + \xi^{18} = (\xi^2 + 1) + (\xi + 1) = \xi^2 + \xi = \xi^{19}$
                    \end{itemize}
                    $2Q = (\xi^{5}, \xi^{19})$
        %%
        %%
        \item[$3Q$] $ = 2Q + Q = (\xi^{5}, \xi^{19}) + (\xi^9, \xi^{18})$
                    \begin{itemize}
                        \item[$\lambda$] $ = (y_2 + y_1)(x_2 + x_1)^{-1} = (\xi^{18} + \xi^{19})(\xi^{5} + \xi^{9})^{-1} = (\xi^4 + \xi^3 + \xi^2 + \xi + 1)(\xi^2 + 1)^{-1} = \xi^{5} (\xi^{15})^{-1} = \xi^{-10} = \xi^{21}$
                        \item[$x_3$] $ = \lambda^2 + \lambda + a + x_1 + x_2 = \xi^{11} + \xi^{21} + \xi^{5} + \xi^{5} + \xi^{9} = \xi^{15} + \xi^9 = \xi^5$
                        \item[$y_3$] $ = \lambda(x_1 + x_3) + x_3 + y_1 = \xi^{21}(\xi^5 + \xi^5) + \xi^5 + \xi^{19} = \xi^{18}$
                    \end{itemize}
                    $3Q = (\xi^{2}, \xi^{18})$
        %%
        %%
        \item[$4Q$] $ = 3Q + Q = (\xi^{5}, \xi^{18}) + (\xi^9, \xi^{18})$
            \begin{itemize}
                \item[$\lambda$] $ = (y_2 + y_1)(x_2 + x_1)^{-1} = (\xi^{18} + \xi^{18})(\xi^{5} + \xi^{9})^{-1} = 0$
                \item[$x_3$] $ = \lambda^2 + \lambda + a + x_1 + x_2 = \xi^{5} + \xi^{5} + \xi^{9} = \xi^{9}$
                \item[$y_3$] $ = \lambda(x_1 + x_3) + x_3 + y_1 = \xi^{9} + \xi^{18} = \xi^{25}$
            \end{itemize}
            $4Q = (\xi^{9}, \xi^{25})$
        %%
        %%
        \item[$5Q$] $ = 4Q + Q = (\xi^{9}, \xi^{25}) + (\xi^9, \xi^{18})$
            \begin{itemize}
                \item[$\lambda$] $ = (y_2 + y_1)(x_2 + x_1)^{-1} = (\xi^{25} + \xi^{18})(\xi^{9} + \xi^{9})^{-1}  = 0$
                \item[$x_3$] $ = \lambda^2 + \lambda + a + x_1 + x_2 = \xi^{5} + \xi^{9} + \xi^{9} = \xi^{5}$
                \item[$y_3$] $ = \lambda(x_1 + x_3) + x_3 + y_1 = \xi^{5} + \xi^{9} = \xi^{15}$
            \end{itemize}
            $5Q = (\xi^{5}, \xi^{15})$
        %%
        %%
        \item[$6Q$] $ = 5Q + Q = (\xi^{5}, \xi^{15}) + (\xi^9, \xi^{18})$
            \begin{itemize}
                \item[$\lambda$] $ = (y_2 + y_1)(x_2 + x_1)^{-1} = (\xi^{15} + \xi^{18})(\xi^{5} + \xi^{9})^{-1} = (\xi^{13})(\xi^{15})^{-1} = \xi^{-2} = \xi^{29}$
                \item[$x_3$] $ = \lambda^2 + \lambda + a + x_1 + x_2 = \xi^{27} + \xi^{29} + \xi^{5} + \xi^{5} + \xi^{9} = \xi + \xi^{9} = \xi^{21}$
                \item[$y_3$] $ = \lambda(x_1 + x_3) + x_3 + y_1 = \xi^{29} (\xi^{5} + \xi^{21}) + \xi^{21} + \xi^{15} = \xi^{29} \xi^{14} + \xi^{11} = \xi^{12} + \xi^{11} = \xi^{29}$
            \end{itemize}
            $6Q = (\xi^{21}, \xi^{29})$
    \end{itemize}

    Los puntos calculados quedan de la siguiente forma: 
    \begin{center}
        \begin{tabular}{| c | c | c |}
            \hline
            0 & 0 & $(0,0)$\\
            1 & $Q$ & $(\xi^9, \xi^{18})$\\
            2 & $2Q$ & $(\xi^{5}, \xi^{19})$\\
            3 & $3Q$ & $(\xi^{2}, \xi^{18})$\\
            4 & $4Q$ & $(\xi^{9}, \xi^{25})$\\
            5 & $5Q$ & $(\xi^{5}, \xi^{15})$\\
            6 & $6Q$ & $(\xi^{21}, \xi^{29})$\\ \hline
        \end{tabular}
    \end{center}

    Procedemos a continuación al cálculo de $-7Q$:

    \begin{itemize}
        \item[$7Q$] $ = 6Q + Q = (\xi^{21}, \xi^{29}) + (\xi^9, \xi^{18})$
            \begin{itemize}
                \item[$\lambda$] $ = (y_2 + y_1)(x_2 + x_1)^{-1} = (\xi^{29} + \xi^{18})(\xi^{21} + \xi^{9})^{-1} = (\xi^{6})(\xi)^{-1} = \xi^{5}$
                \item[$x_3$] $ = \lambda^2 + \lambda + a + x_1 + x_2 = \xi^{10} + \xi^{5} + \xi^{5} + \xi^{21} + \xi^{9} = \xi^{29} + \xi^{9} = \xi^{17}$
                \item[$y_3$] $ = \lambda(x_1 + x_3) + x_3 + y_1 = \xi^{5} (\xi^{21} + \xi^{17}) + \xi^{17} + \xi^{29} = \xi^{5} \xi^{27} + \xi^{9} = \xi + \xi^{9} = \xi^{21}$
            \end{itemize}
            $7Q = (\xi^{17}, \xi^{21})$
            \begin{itemize}
                \item[$\Rightarrow$] $-7Q = (x_3, x_3 + y_3) = (\xi^{17}, \xi^{27})$
            \end{itemize}
        \item[$2(-7Q)$] $ = (-7Q) + (-7Q) = (\xi^{17}, \xi^{27}) + (\xi^{17}, \xi^{27})$
            \begin{itemize}
                \item[$\lambda$] $ = x_1 + y_1x_1^{-1} = \xi^{17} + \xi^{27} (\xi^{17})^{-1} = \xi^{17} + \xi^{10} = \xi$
                \item[$x_3$] $ = \lambda^2 + \lambda + a + x_1 + x_2 = \xi^{2} + \xi + \xi^{5} = \xi^{18}$
                \item[$y_3$] $ = \lambda(x_1 + x_3) + x_3 + y_1 = \xi (\xi^{17} + \xi^{18}) + \xi^{18} + \xi^{27} = \xi^{5} + \xi^{3} = \xi^{8}$
            \end{itemize}
            $2(-7Q) = (\xi^{18}, \xi^{8})$
        \item[$3(-7Q)$] $ = 2(-7Q) + (-7Q) = (\xi^{18}, \xi^{8}) + (\xi^{17}, \xi^{27})$
            \begin{itemize}
                \item[$\lambda$] $ = (y_2 + y_1)(x_2 + x_1)^{-1} = (\xi^{8} + \xi^{27}) (\xi^{17} + \xi^{18})^{-1} = \xi^{19} (\xi^{4})^{-1} = \xi^{15}$
                \item[$x_3$] $ = \lambda^2 + \lambda + a + x_1 + x_2 = \xi^{30} + \xi^{15} + \xi^{5} + \xi^{18} + \xi^{17} = \xi^{8} + \xi^{5} + \xi^{4} = \xi^{21}$
                \item[$y_3$] $ = \lambda(x_1 + x_3) + x_3 + y_1 = \xi^{15} (\xi^{18} + \xi^{21}) + \xi^{21} + \xi^{8} = \xi^{15}\xi^{16} + \xi^{22} = \xi^{7}$
            \end{itemize}
            $3(-7Q) = (\xi^{21}, \xi^{7})$
        \item[$4(-7Q)$] $ = 3(-7Q) + (-7Q) = (\xi^{21}, \xi^{7}) + (\xi^{17}, \xi^{27})$
            \begin{itemize}
                \item[$\lambda$] $ = (y_2 + y_1)(x_2 + x_1)^{-1} = (\xi^{7} + \xi^{27}) (\xi^{17} + \xi^{21})^{-1} = \xi^{15} (\xi^{27})^{-1} = \xi^{-12} = \xi^{19}$
                \item[$x_3$] $ = \lambda^2 + \lambda + a + x_1 + x_2 = \xi^{7} + \xi^{19} + \xi^{5} + \xi^{21} + \xi^{17} = \xi^{30} + \xi^{5} + \xi^{27} = \xi^{26} + \xi^{27} = \xi^{13}$
                \item[$y_3$] $ = \lambda(x_1 + x_3) + x_3 + y_1 = \xi^{19} (\xi^{21} + \xi^{13}) + \xi^{13} + \xi^{7} = \xi^{21} + \xi^{3} = \xi^{4}$
            \end{itemize}
            $4(-7Q) = (\xi^{13}, \xi^{4})$
        \item[$5(-7Q)$] $ = 4(-7Q) + (-7Q) = (\xi^{13}, \xi^{4}) + (\xi^{17}, \xi^{27})$
            \begin{itemize}
                \item[$\lambda$] $ = (y_2 + y_1)(x_2 + x_1)^{-1} = (\xi^{4} + \xi^{27}) (\xi^{17} + \xi^{13})^{-1} = \xi^{16} (\xi^{23})^{-1} = \xi^{-7} = \xi^{24}$
                \item[$x_3$] $ = \lambda^2 + \lambda + a + x_1 + x_2 = \xi^{17} + \xi^{24} + \xi^{5} + \xi^{17} + \xi^{13} = \xi^{8} + \xi^{5} + \xi^{23} = \xi^{11}$
                \item[$y_3$] $ = \lambda(x_1 + x_3) + x_3 + y_1 = $
            \end{itemize}
            $5(-7Q) = ()$

    \end{itemize}



    \newpage
    \textbf{Ejercicio 2. }\textit{Para tu curva y tu punto base, genera un par de claves pública/privada para el protocolo ECDH.}
    \vspace{0.5cm}

    Tomamos una clave privada que llamaremos $c$ a partir de la cual generaremos una clave pública dada por $(E, Q, cQ)$, siendo $E$ la curva
    y $Q$ el punto base considerados en el ejercicio anterior. Tomamos como clave privada $c = 2$ para simplificar los cálculos (aunque dicho valor puede ser aleatorio), 
    y obtenemos la clave pública 
    $$(E, Q, cQ) = ((\xi^5, \xi^{16}), (\xi^9, \xi^{18}), 2Q) = ((\xi^5, \xi^{16}), (\xi^9, \xi^{18}), (\xi^{5}, \xi^{19}))$$


    \newpage
    \textbf{Ejercicio 3. }\textit{Cifra el mensaje $(\xi^3 + \xi^2 + 1, \xi^4 + \xi^2)$ mediante el criptosistema de Menezes-Vanstone.}
    \vspace{0.5cm}

    Tomemos, además de la clave privada anterior $c$, un nuevo valor para $k$, el cual tomaremos como $k = 2$. Calculamos el punto $(x_0, y_0)$ que viene dado por 
    $$ (x_0, y_0) = a \cdot k \cdot Q = 4Q = (\xi^{9}, \xi^{25})$$

    Tomamos el mensaje $(m_1, m_2) = (\xi^3 + \xi^2 + 1, \xi^4 + \xi^2) = (\xi^{8}, \xi^{7})$ y procedemos al cifrado mediante el algoritmo de Menezes-Vanstone:
    $$ E(m_1, m_2) = (kQ, x_0 m_1, y_0 m_2) = (2Q, \xi^{9} \xi^{8}, \xi^{25} \xi^{7}) = ((\xi^{5}, \xi^{19}), \xi^{17}, \xi)$$

    \newpage
    \textbf{Ejercicio 4. }\textit{Descifra el mensaje anterior.}
    \vspace{0.5cm}


\end{document}