\documentclass[fleqn]{article}

%\pgfplotsset{compat=1.17}

\usepackage{mathexam}
\usepackage{amsmath}
\usepackage{amsfonts}
\usepackage{graphicx}
\usepackage{systeme}
\usepackage{microtype}
\usepackage{multirow}
\usepackage{pgfplots}
\usepackage{listings}
\usepackage{tikz}
\usepackage{dsfont} %Numeros reales, naturales...
\usepackage{cancel}
\usepackage{babel}

%\graphicspath{{images, }}
\newcommand*{\QED}{\hfill\ensuremath{\square}}

%Estructura de ecuaciones
\setlength{\textwidth}{15cm} \setlength{\oddsidemargin}{5mm}
\setlength{\textheight}{23cm} \setlength{\topmargin}{-1cm}



\author{David García Curbelo}
\title{Criptografía}

\pagestyle{empty}


\def\R{\mathds{R}}
\def\Z{\mathds{Z}}
\def\N{\mathds{N}}
\def\Q{\mathds{Q}}

\def\sup{$^2$}

\def\next{\quad \Rightarrow \quad}


\begin{document}
    \begin{center}
        \LARGE{\textbf{Ejercicio 2}} \\
        \Large{David García Curbelo} \\
    \end{center}

    \vspace{1cm}

    Partimos de nuestro dni$= 45352581$. Dividimos dicho número en dos bloques, 4535 y 2581. Sean $p = 4547$ y $q = 2591$ los 
    primeros primos mayores o iguales que los bloques anteriores. Sea $n = pq = 11781277$ y $e$ el menor primo mayor o igual que
    11 que es primo relativo con $\varphi(n)$. Sea $d = e^{-1} \pmod{\varphi(n)}$.

    Tenemos así $\varphi(n) = 11774140$, $e = 11$ y $d = 8563011$.

    \vspace{1cm}

    \textbf{Apartado I. \textit{Cifra el mensaje $m = 0xCAFE$.}}\\
    Usamos la siguiente función de cifrado:
    $$\text{RSA}_{n,e} (m) = m^e \pmod{n}$$
    Pasando $m$ a decimal y aplicando la función anterior, obtenemos el siguiente resultado:
    $$text{RSA}_{11781277, 11} (51966) = 51966^{11} \pmod{11781277} = 9088323 \pmod{11781277}$$
    Donde el mensaje queda cifrado de la siguente manera: $9088323 = 0x8AAD43$.


    \newpage
    \textbf{Apartado II. \textit{Descifra el criptograma anterior.}}\\
    Procedamos a continuación a descifrar el criptograma $c = 0x8AAD43 = 9088323 \pmod{11781277}$. Usamos la siguiente función de 
    descifrado:
    $$\text{RSA}_{n,e}^{-1} (c) = c^d \pmod{n}$$
    Por tanto procedemos a descifrar el mensaje pedido y el criptograma queda:
    $$text{RSA}_{11781277, 11}^{-1} (9088323) = 9088323^{8563011} \pmod{11781277} = 51966 \pmod{11781277}$$
    Donde $51966 = 0xCAFE$ es el mensaje que buscábamos.


    \newpage
    \textbf{Apartado III. \textit{Intenta factorizar $n$ mediante el método $P-1$ de Polard. Para ello llega, como máximo a $b=8$.}}\\
    Tratemos de factorizar $n$ en 8 pasos como máximo. Para ello, vamos a utilizar el método $P-1$ de Polard. Usaremos como base el 2, 
    que sabemos que es primo relativo con $n$. Comencemos con las iteraciones.
    \begin{enumerate}
        \item[$b=1$] $$2^{1!} \equiv 2 \pmod{n}, \quad \text{con }\text{ mcd}(2-1,n) = 1$$
        \item[$b=2$] $$2^{2!} \equiv 4 \pmod{n}, \quad \text{con }\text{ mcd}(4-1,n) = 1$$
        \item[$b=3$] $$2^{3!} \equiv 64 \pmod{n}, \quad \text{con }\text{ mcd}(64-1,n) = 1$$
        \item[$b=4$] $$2^{4!} \equiv 4995939 \pmod{n}, \quad \text{con }\text{ mcd}(4995939-1,n) = 1$$
        \item[$b=5$] $$2^{5!} \equiv 3564251 \pmod{n}, \quad \text{con }\text{ mcd}(3564251-1,n) = 1$$
        \item[$b=6$] $$2^{6!} \equiv 1811135 \pmod{n}, \quad \text{con }\text{ mcd}(1811135-1,n) = 1$$
        \item[$b=7$] $$2^{7!} \equiv 9030003 \pmod{n}, \quad \text{con }\text{ mcd}(9030003-1,n) = 1$$
        \item[$b=8$] $$2^{8!} \equiv 9730811 \pmod{n}, \quad \text{con }\text{ mcd}(9730811-1,n) = 1$$
    \end{enumerate}

    En este caso no ha sido posible factorizar $n$ en 8 pasos.


    \newpage
    \textbf{Apartado IV. \textit{Intenta factorizar $n$ a partir de $\varphi(n)$.}}\\
    Hayar una factorización en nuestro caso consiste en obtener $p$ y $q$ (ya que $n = pq$). Para ello, tenemos la siguiente ecuación:
    $$(x-p)(x-q) = x^2 - (n + 1 - \varphi(n)) \cdot x - n$$

    Tratemos de resolverla. Sabemos que $p$ y $q$ son las soluciones de $b \pm \sqrt{b^2 -n}$, donde
    sabemos que $p + q = 2b = n - \varphi(n) + 1 = 7138$, luego obtenemos que $b = 3569$. Ahora resolviendo tenemos
    $$b \pm \sqrt{b^2 -n} = 3569 \pm \sqrt{956484} = 3569 \pm 978 \next p = 4547, \quad q = 2591$$

    \newpage
    \textbf{Apartado V. \textit{Intenta factorizar $n$ a partir de $e$ y $d$.}}\\ 
    Partimos de $n = 11781277$, $e = 11$ y $d = 8563011$. Tratemos de factorizar $n$. Para ello calculemos previamente 
    $k = e \cdot d - 1 = 94193120$ y por tanto $k/2 = 47096560$. Calculamos ahora
    $a^{k/2^i} \pmod{n}$ para $i = 1, 2, 3, 4, 5$ y $a = 2, 3, 5, 7, 11$.
    \begin{equation*}k/2: \left\{\\
        \begin{aligned}
            &2^{47096560} \pmod{11781277} = 1 \\
            &3^{47096560} \pmod{11781277} = 1 \\
            &5^{47096560} \pmod{11781277} = 1 \\
            &7^{47096560} \pmod{11781277} = 1 \\
            &11^{47096560} \pmod{11781277} = 1 \\
        \end{aligned} \right.
    \end{equation*}
    \begin{equation*}
        k/4: \left\{ \\
        \begin{aligned}
            &2^{23548280} \pmod{11781277} = 1 \\
            &3^{23548280} \pmod{11781277} = 1 \\
            &5^{23548280} \pmod{11781277} = 1 \\
            &7^{23548280} \pmod{11781277} = 1 \\
            &11^{23548280} \pmod{11781277} = 1 \\
        \end{aligned} \right.
    \end{equation*}
    \begin{equation*}
        k/8: \left\{ \\
        \begin{aligned}
            &2^{11774140} \pmod{11781277} = 1 \\
            &3^{11774140} \pmod{11781277} = 1 \\
            &5^{11774140} \pmod{11781277} = 1 \\
            &7^{11774140} \pmod{11781277} = 1 \\
            &11^{11774140} \pmod{11781277} = 1 \\
        \end{aligned} \right. 
    \end{equation*}
    \begin{equation*}
        k/16: \left\{ \\
        \begin{aligned}
            &2^{5887070} \pmod{11781277} = 1 \\
            &3^{5887070} \pmod{11781277} = 1 \\
            &5^{5887070} \pmod{11781277} = 1 \\
            &7^{5887070} \pmod{11781277} = 1 \\
            &11^{5887070} \pmod{11781277} = 1 \\
        \end{aligned} \right.
    \end{equation*}
    \begin{equation*}
        k/32: \left\{ \\
        \begin{aligned}
            &2^{2943535} \pmod{11781277} = 2541772 \\
        \end{aligned} \right.
    \end{equation*}

    Hemos obtenido que $2^{k/32} \pmod{n} = 2541772 \neq 1$, por tanto obtenemos la factorización como sigue:
    \begin{equation*}
        p = \text{mcd} (n, 2541772 + 1) = 4547 \\
        q = \text{mcd} (n, 2541772 - 1) = 2591 
    \end{equation*}

\end{document}