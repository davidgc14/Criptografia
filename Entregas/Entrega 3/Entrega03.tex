\documentclass[fleqn]{article}

%\pgfplotsset{compat=1.17}

\usepackage{mathexam}
\usepackage{amsmath}
\usepackage{amsfonts}
\usepackage{graphicx}
\usepackage{systeme}
\usepackage{microtype}
\usepackage{multirow}
\usepackage{pgfplots}
\usepackage{listings}
\usepackage{tikz}
\usepackage{dsfont} %Numeros reales, naturales...
\usepackage{cancel}
\usepackage{babel}

%\graphicspath{{images, }}
\newcommand*{\QED}{\hfill\ensuremath{\square}}

%Estructura de ecuaciones
\setlength{\textwidth}{15cm} \setlength{\oddsidemargin}{5mm}
\setlength{\textheight}{23cm} \setlength{\topmargin}{-1cm}



\author{David García Curbelo}
\title{Criptografía}

\pagestyle{empty}


\def\R{\mathds{R}}
\def\Z{\mathds{Z}}
\def\N{\mathds{N}}
\def\Q{\mathds{Q}}
\def\F{\mathds{F}}

\def\sup{$^2$}

\def\next{\quad \Rightarrow \quad}


\begin{document}
    \begin{center}
        \LARGE{\textbf{Ejercicio 3}} \\
        \Large{David García Curbelo} 
    \end{center}

    \vspace{1cm}     

    \textit{Los parámetros de un criptosistema de ElGamal son $p = 211$ y 
    $g = 3$, es decir, el criptosistema está diseñado en el cuerpo $\F_{211} = \Z_{211}$ y tomamos
    como generador de $\F_{211}^*$, $g = 3$. La clave pública empleada es $3^a = 109 \pmod{211}$.
    Descifra el criptograma $(154, \text{dni} \pmod{211})$, donde dni es el número de tu DNI. Para 
    calcular los logaritmos discretos necesarios emplea dos de los métodos descritos en la teoría.} 

    \vspace{0.5cm}   
        
    Por el enunciado y por mi $DNI = 45352581$, obtenemos el criptograma $(154, \text{dni} \pmod{211}) 
    = (154, 30)$. Procedemos a aplicar los dos algoritmos vistos en teoría:
    \begin{center}
        \large{\textbf{Paso de Bebé - Paso de Gigante}}  
    \end{center}

    Sabemos que $D_\alpha (x, y) = y \cdot x^{-\alpha}$, por lo que vamos a calcularlo mediante el 
    logaritmo discreto. Por el enunciado, sabemos que la clave pública viene dada por $3^\alpha = 109 \pmod{211}$,
    por lo que tenemos el problema $\alpha = \log_3 109 \pmod{211}$. Además conocemos la siguiente información:
    \begin{itemize}
        \item $G = \F_{211}^*$
        \item $g = 3$
        \item $h = 109$
        \item $f = [\sqrt{p-1}] = [\sqrt{210}] = 15$
    \end{itemize}

    Procedemos ahora a la construcción de la tabla de iteraciones:

    \begin{center}
        \begin{tabular}{|c c c c c c c c c c c c c c c|} \hline
            0 & 1 & 2 & 3  & 4  & 5  & 6  & 7  & 8  & 9  & 10  & 11  & 12  & 13 & 14 \\ \hline
            1 & 3 & 9 & 27 & 81 & 32 & 96 & 77 & 20 & 60 & 180 & 118 & 143 & 17 & 21 \\ \hline
        \end{tabular}
    \end{center}

    Además tenemos $$g^{-f} = 3^{-15} = 3^{211 - 1 - 15} = 3^195 = 67 \pmod{211}$$

    \begin{itemize}
        \item $h_0 = 109$ no pertenece a ninguna de las iteraciones de la tabla.
        \item $h_1 = 109 \cdot 67 = 129$ no pertenece a ninguna de las iteraciones de la tabla.
        \item $h_2 = 129 \cdot 67 = 203$ no pertenece a ninguna de las iteraciones de la tabla.
        \item $h_3 = 203 \cdot 67 = 97$ no pertenece a ninguna de las iteraciones de la tabla.
        \item $h_4 = 97 \cdot 67 = 169$ no pertenece a ninguna de las iteraciones de la tabla.
        \item $h_5 = 169 \cdot 67 = 140$ no pertenece a ninguna de las iteraciones de la tabla.
        \item $h_6 = 140 \cdot 67 = 96$ sí pertenece a la tabla, concretamente en la sexta iteración.
    \end{itemize}

    Hemos encontrado un $h_k$ que pertenece a la tabla, obteniendo $i = 6$ y $j = 6$. Obtenemos por tanto el resultado
    que andábamos buscando, de la forma $\alpha = \log_3 109 \pmod{211} = 6 + 6 \cdot 15 = 96$.
    \begin{flalign*}
        D_{96} (154, 30) &= 30 \cdot 154^{p - 1 - \alpha} \pmod{211}\\
                        &= 30 \cdot 154^{211 - 1 - 96} \pmod{211}\\
                        &= 30 \cdot 154^{114} \pmod{211}\\
                        &= 30 \cdot 114 \pmod{211}\\
                        &= 44 \pmod{211}
    \end{flalign*}

    Por tanto hemos obtenido el mensaje, el cual es $m = 44$.
    \newpage

    \begin{center}
        \large{\textbf{Silver - Pohlig - Hellman}}  
    \end{center}

    Consideremos de nuevo $p = 211$, $h = 109$ y $g = 3$ el generador del grupo $\F_{211}^*$. Para conocer el número de 
    iteraciones necesarias, factorizamos $p - 1 = 210 = 2 \cdot 3 \cdot 5 \cdot 7$, lo que nos informa que el algoritmo
    necesita de 4 iteraciones.

    \begin{enumerate}
        \item $ p_1 = 2, e_1 = 1 \next p_1^{e_1} = 2 \\
                r_0 = 1 \\
                r_1 = 3^{(1 \cdot 210)/2} \pmod{211} = 210 \\
                y_0 = h = 109 \\
                109^{210/2} = 1 \pmod{211} \next x_0 = 0 \\
                m = 0 \pmod{2} $
        \item
    \end{enumerate}

\end{document}