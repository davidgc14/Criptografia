\documentclass[fleqn]{article}

%\pgfplotsset{compat=1.17}

\usepackage{mathexam}
\usepackage{amsmath}
\usepackage{amsfonts}
\usepackage{graphicx}
\usepackage{systeme}
\usepackage{microtype}
\usepackage{multirow}
\usepackage{pgfplots}
\usepackage{listings}
\usepackage{tikz}
\usepackage{dsfont} %Numeros reales, naturales...
\usepackage{cancel}
\usepackage{babel}

%\graphicspath{{images, }}
\newcommand*{\QED}{\hfill\ensuremath{\square}}

%Estructura de ecuaciones
\setlength{\textwidth}{15cm} \setlength{\oddsidemargin}{5mm}
\setlength{\textheight}{23cm} \setlength{\topmargin}{-1cm}



\author{David García Curbelo}
\title{Criptografía}

\pagestyle{empty}


\def\R{\mathds{R}}
\def\Z{\mathds{Z}}
\def\N{\mathds{N}}
\def\Q{\mathds{Q}}
\def\F{\mathds{F}}

\def\sup{$^2$}

\def\next{\qquad \Rightarrow \qquad}


\begin{document}

    \section{RSA}
    \begin{itemize}
        \item $p = 11, \thinspace q = 7, \thinspace d = 53 \next (77, 17)$
        \item $(119, 5), \thinspace E = 81 \next m = 30$
        \item $(65, 7), \thinspace E = 31 \next m = 21$
        \item $(299, 5) \next p = 13, \thinspace q = 23, \thinspace d = 53$
        \item No se puede afirmar que calcular $d$ a partir de $(n, e)$ sea polinomial.
    \end{itemize}

    \section{Curvas Elípticas}
    \begin{itemize}
        \item Si tomamos una c.e. módulo $p$ dada en forma de Weierstrass $\next$ El número de puntos de la c.e. está comprendido en el intervalo $[(\sqrt{p} - 1)^2, (\sqrt{p} + 1)^2]$.
        \item (V/F) Una c.e. sobre un cuerpo $K$ tiene siempre un punto proyectivo con coordenadas enteras.
        \item (V/F) Tres puntos alineados de una c.e. siempre suman cero.
        \item Si $E(\F_{p^k})$ es el grupo de una c.e. $\next$ Si es cíclico, no puede tener más de un elemento de orden dos.
        \item Sea $\F_{2^{233}}$. ¿Por qué no es bueno emplear la ecuación $y^2 + (\xi^{221} + \xi^{120})y = x^3 + x + \xi^{3122}$? $\next$ Porque es supersingular.
    \end{itemize}
    \subsection{Menezes-Vanstone}
    \begin{itemize}
        \item Sobre $\F_{11}$ utilizamos la curva $y^2 = x^3 + 2x + 5$ y el punto $Q = (9,2)$, clave privada $a = 3$ y su clave pública $aQ = (8,4)$. ¿Cuál NO puede ser cifrado del mensaje $(5,5)$? $\next ((9,2), 9, 7)$
        \item Sobre $\F_{41}$ utilizamos la curva $y^2 = x^3 + 33x + 35$ y el punto $Q = (8,27)$, clave privada $a = 21$ y su clave pública $aQ = (6,3)$. ¿Cuál SI puede ser cifrado del mensaje $(32,22)$? $\next ((19,10), 35, 33)$
        \item Sobre $\F_{16}$ utilizamos la curva $y^2 + xy = x^3 + (\xi^3 + \xi + 1)x^2 + \xi^3 + \xi$ y el punto $Q = (\xi^3 + \xi^2 + \xi + 1,\xi^2 + 1)$, clave privada $a = 3$ y su clave pública $aQ = (\xi^3 + \xi^2 + \xi, \xi^3 + \xi^2)$. 
                Obtenemos el criptograma $((\xi^3 + \xi^2 + \xi + 1, \xi^3 + \xi), \xi^3 + 1, \xi^3 + \xi + 1)$ ¿Cuál es el mensaje? $\next (\xi^3, \xi^3 + \xi^2)$
        \item Sobre $\F_{17}$ utilizamos la curva $y^2 = x^3 + x + 1$ y el punto $Q = (0,1)$, clave privada de B es $a = 3$ y la clave pública de A es $aQ = (15,5)$. Clave compartida: $\next (10,5)$
    \end{itemize}

    \section{Primos de Fermat}
    \begin{itemize}
        \item (V/F) Un pseudoprimo de Fermat, $n = psp(a)$, satisface $a^{n-1} \equiv 1 \pmod{n}$ y es compuesto.
        \item (V/F) Los pseudoprimos fuertes pueden certificar que un número es compuesto pero no que es primo.
        \item (V/F) Aunque sea fácil comprobar la primalidad de un número de Fermat puede ser dificil demostrar la primalidad de alguno de sus factores.
        \item (V/F) Un pseudoprimo de Euler respecto de la base $a$ es siempre pseudoprimo de Fermat respecto de la misma base.
        \item (V/F) Sólo se conocen un número finito de números de Carmichel.
        \item (V/F) Los tests de Solovay-Strassen y el de Miller-Rabin pueden certificar que un número es compuesto.
    \end{itemize}

    \section{FCS}
    \begin{itemize}
        \item La FCS de $\sqrt{d}$ con $d$ libre de cuadrados es $[q_0, \dots, 2q_0]$ donde cada $q_i < q_0$.
        \item Si $\alpha = \frac{P + \sqrt{d}}{Q}$ es un irracional cuadrático ($d$ libre de cuadrados): 
                \begin{itemize}
                    \item La FCS de $\alpha$ es periódica con periodo máximo $2d - 1$.
                    \item La FCS de $\alpha$ es puramente periódica sii $\alpha > 1$ y $1 < \overline{\alpha} < 0$ (su conjugado).
                \end{itemize}
        \item (V/F) Una FCS finita coincide con su último convergente.
        \item Si $x^2 - dy^2 = N$ ($|N| < \sqrt{d}$) es una ecuación de Pell $\next$ Cualquier solución positiva con $mcd(x,y) = 1$, son el numerador y el denominador de una convergente de la FCS de $\sqrt{d}$.
        \item $\alpha = \sqrt{2} \next \alpha = [1,2,2,\dots]$
        \item $\alpha = \sqrt{3} \next $ No es puramente periódica.
        \item $\alpha = \frac{a + \sqrt{a^2 + 4}}{2} \next \alpha = [a,a,a,\dots]$
        \item $\alpha = \frac{1 + \sqrt{5}}{2} \next \alpha = [1,1,1,\dots]$
        \item (V/F) $\sqrt{a^2 - 1} = [a-1, \overline{1, 2(a-1), \dots}]$
    \end{itemize}

    \section{Diffie-Hellman}
    \begin{itemize}
        \item $p = 73$, $g = 5$. Claves públicas $A = (p, q, 37)$ y $B = (p, q, 12)$. Clave compartida $\quad \Rightarrow \quad$ 32.
        \item $p = 37$, $g = 2$. Clave pública $A = (p, q, 3)$ y clave compartida 10. Clave privada B $\quad \Rightarrow \quad$ 30.
    \end{itemize}

    \section{ElGamal}
    \begin{itemize}
        \item Cuál de las siguientes parejas NO puede ser el cifrado de $m = 10$ con ElGamal y clave privada $a = 4$ con parámetros:
                \begin{itemize}
                    \item $p = 23, \thinspace g = 5 \next 14,7$
                    \item $p = 17, \thinspace g = 3 \next 14,7$
                \end{itemize}
    \end{itemize}

    \section{Logaritmo Discreto}
    \begin{itemize}
        \item ¿Cual de los algoritmos NO se puede emplear para el cálculo de log. dis. en $\F_{2^{1024}}$? $\next $ Cálculo de índice en cuerpos primos.
        \item Aplicamos Silver-Pohlig-Hellman para el cálculo de log. dis. de un elemento $b$ de orden $n = 700$. ¿Cuántas raíces de la unidad en $<b>$ hay que calcular? $\next 14$.
    \end{itemize}

    \section{Raíces cuadradas}
    \begin{itemize}
        \item ¿Qué $\gamma$ puede usarse para calcular la raíz de 27 módulo 37? $\next \gamma = 18$. 
        \item ¿Cuál de esto enteros no tiene raíz cuadrada módulo 53? $\next 30$.
    \end{itemize}

    \section{Modos}
    \begin{itemize}
        \item ECB. El que menos bits cambia en el criptograma cuando alteramos un bit en el mensaje.
        \item CBC. Ninguna de las otras opciones.
        \item OFB. Convierte un cifrado de bloque en un cifrado de flujo síncrono.
        \item CFB. Convierte un cifrado de bloque en un cifrado de flujo autosincronizable.
    \end{itemize}

    \section{Teoría}
    \begin{itemize}
        \item Integridad: La información no ha sido alterada en el envío.
        \item No repudio: El emisario no puede negar haber realizado el envío.
        \item Autenticidad: La información proviene de quien dice enviarla.
        \item Confidencialidad: La información sólo puede ser accesible por las entidades autorizadas.
        \item Cifrado de flujo síncrono. Es más vulnerable que un cifrado de flujo autosincronizable al cambio de un carácter en el criptograma.
        \item Criptosistema de clave pública. No podemos $\next$ ninguna de las tres opciones.
    \end{itemize}

    \section{Preguntas sueltas}
    \begin{itemize}
        \item[\textbf{BABY STEP}] Sea $G$ un grupo, $b \in G$ de orden 101. en BS-GS el número máximo de elementos de $G$ que necesitamos tener almacenados es $\next 13$.
        \item[\textbf{PRATT}]  (V/F) El certificado de Pratt es recursivo y usa el concepto de orden multiplicativo módulo $n$.
        \item[\textbf{MONTECARLO}] (V/F) El test probabilístico tipo Montecarlo corre en tiempo polinomial y puede ser inclinado a TRUE, a FALSE o no inclinado.
        \item[\textbf{LUCAS-LEHMER}] (V/F) Existe un $n$ no primo con el grupo multiplicativo de las unidades módulo $n$ cíclico.
        \item[\textbf{EC. CUADRÁTICA}] ¿Cuál en $\F_{32}$ tiene solución? $\next z^2 + (\xi^3 + \xi)z + (\xi^3 + \xi^2)$ 
        \item[\textbf{LAS VEGAS}] (V/F) Un test probabilístico tipo Las Vegas produce una respuesta correcta en tiempo aleatorio cuya media está acotada polinomialmente.
    \end{itemize}




\end{document}