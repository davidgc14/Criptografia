\documentclass[fleqn]{article}

%\pgfplotsset{compat=1.17}

\usepackage{mathexam}
\usepackage{amsmath}
\usepackage{amsfonts}
\usepackage{graphicx}
\usepackage{systeme}
\usepackage{microtype}
\usepackage{multirow}
\usepackage{pgfplots}
\usepackage{listings}
\usepackage{tikz}
\usepackage{dsfont} %Numeros reales, naturales...
\usepackage{cancel}
\usepackage{babel}

%\graphicspath{{images, }}
\newcommand*{\QED}{\hfill\ensuremath{\square}}

%Estructura de ecuaciones
\setlength{\textwidth}{15cm} \setlength{\oddsidemargin}{5mm}
\setlength{\textheight}{23cm} \setlength{\topmargin}{-1cm}



\author{David García Curbelo}
\title{Criptografía}

\pagestyle{empty}


\def\R{\mathds{R}}
\def\Z{\mathds{Z}}
\def\N{\mathds{N}}
\def\Q{\mathds{Q}}

\def\sup{$^2$}

\def\next{\quad \Rightarrow \quad}


\begin{document}
    \begin{center}
        \LARGE{\textbf{Ejercicio 1}} \\
        \Large{David García Curbelo} \\
    \end{center}

    \vspace{1cm}

    Consideremos el cifrado por bloques miniAES descrito en el ejercicio 2.1.


    \textbf{Apartado I. \textit{Calcula $\text{E}_{\text{dni}}(0x01234567)$ usando el modo $CBC$ e $IV = 0x0001$.}}\\


    
    \textbf{Apartado I. \textit{Calcula $\text{E}_{\text{dni}}(0x01234567)$ usando el modo $CFB$, $r=11$, y vector de 
            inicialización $IV = 0x0001$.}}\\


\end{document}