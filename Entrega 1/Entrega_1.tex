\documentclass[fleqn]{article}

%\pgfplotsset{compat=1.17}

\usepackage{mathexam}
\usepackage{amsmath}
\usepackage{amsfonts}
\usepackage{graphicx}
\usepackage{systeme}
\usepackage{microtype}
\usepackage{multirow}
\usepackage{pgfplots}
\usepackage{listings}
\usepackage{tikz}
\usepackage{dsfont} %Numeros reales, naturales...
\usepackage{cancel}
\usepackage{babel}

%\graphicspath{{images, }}
\newcommand*{\QED}{\hfill\ensuremath{\square}}

%Estructura de ecuaciones
\setlength{\textwidth}{15cm} \setlength{\oddsidemargin}{5mm}
\setlength{\textheight}{23cm} \setlength{\topmargin}{-1cm}



\author{David García Curbelo}
\title{Criptografía}

\pagestyle{empty}


\def\R{\mathds{R}}
\def\Z{\mathds{Z}}
\def\N{\mathds{N}}
\def\Q{\mathds{Q}}

\def\sup{$^2$}

\def\next{\quad \Rightarrow \quad}


\begin{document}
    \begin{center}
        \LARGE{\textbf{Ejercicio 1}} \\
        \Large{David García Curbelo} \\
    \end{center}

    \vspace{1cm}

    Consideremos el cifrado por bloques miniAES descrito en el ejercicio 2.1.


    \textbf{Apartado I. \textit{Calcula $\text{E}_{\text{dni}}(0x01234567)$ usando el modo $CBC$ e $IV = 0x0001$.}}\\

    Tenemos por dni el número 45352581, luego obtenemos la siguiente clave:
    $$dni \equiv 1669 \pmod(65536) \next clave=1669$$
    y el mensaje que queremos cifrar es el número $0x01234567 = [1,1,1,0,0,1,1,0,1,0,1,0,0,0,1,0,1,1,0,0,0,1,0,0,1]$.
    
    Vamos a calcular el criptograma usando el modo CBC y el cifrado de bloques miniAES. Vamos a dividir nuestro mensaje 
    en dos bloques de 16 bits y así calcular $c_1$ y $c_2$ para cada uno de los bloques.
    $$[\overbrace{1,1,1,0,0,1,1,0,1,0,1,0,0,0,1,0}^{m_1},\overbrace{1,1,0,0,0,1,0,0,1,0,0,0,0,0,0,0}^{m_2}] $$

    Además, por el enunciado tenemos que $c_0 = 0x0001$, y tenemos por tanto que $E_{\text{dni}}(0x01234567) = c_0 c_1 c_2$.
    Tomemos por tanto nuestra clave $k = 1669 = 0x0685 = 0b11010000101$, donde podemos ver que $k_0 = 0$, $k_1 = 6$, $k_2 = 8$ 
    y $k_3 = 5$.

    Función de sustitución $\gamma$ de manera explícita: \\
    $\gamma(0000) = 0011$ \\
    $\gamma(0001) = 1000$ \\	
    $\gamma(0010) = 1111$ \\
    $\gamma(0011) = 0111$ \\
    $\gamma(0100) = 0001$ \\
    $\gamma(0101) = 0010$ \\
    $\gamma(0110) = 1011$ \\
    $\gamma(0111) = 0000$ \\
    $\gamma(1000) = 1100$ \\
    $\gamma(1001) = 1110$ \\
    $\gamma(1010) = 1010$ \\
    $\gamma(1011) = 0110$ \\
    $\gamma(1100) = 1001$ \\
    $\gamma(1101) = 1101$ \\
    $\gamma(1110) = 0101$ \\
    $\gamma(1111) = 0100$ \\
    
    Y obtenemos así las claves de ronda:
    \begin{itemize}
        \item $w_0 = k_0 = 0 = 0000$
        \item $w_1 = k_1 = 6 = 0110$
        \item $w_2 = k_2 = 8 = 1000$
        \item $w_3 = k_3 = 5 = 0101$
        \item $w_4 = w_0 \oplus \gamma(w_3) \oplus 0001 = 0000 \oplus 0010 \oplus 0001 = \alpha + 1 = 0011$
        \item $w_5 = w_1 \oplus w_4 = \alpha^2 + \alpha + \alpha + 1 = \alpha^2 + 1 = 0110$
        \item $w_6 = w_2 \oplus w_5 = \alpha^3 + \alpha^2 + \alpha = 1110$
        \item $w_7 = w_3 \oplus w_6 = \alpha^2 + 1 + \alpha^3 + \alpha^2 + \alpha = \alpha^3 + \alpha + 1 = 1011$
        \item $w_8 = w_4 \oplus \gamma(w_7) \oplus 0010 = \alpha + 1 + \alpha^2 + \alpha + \alpha = \alpha^2 + \alpha + 1 = 0111$
        \item $w_9 = w_5 \oplus w_8 = \alpha^2 + \alpha + \alpha^2 + \alpha + 1 = 1 = 0001$
        \item $w_{10} = w_6 \oplus w_9 = \alpha^3 + \alpha^2 + \alpha + 1 = 1111$
        \item $w_{11} = w_7 \oplus w_{10} = \alpha^3 + \alpha + 1 + \alpha^3 + \alpha^2 + \alpha + 1 = \alpha^2 = 0100$
    \end{itemize}

    Para $c_1$ calculamos ahora $E_k(m_1 \oplus c_0) = E_k(1110011010100011)$. Para ello apliquemos cada una de las funciones de 
    su descomposición $E_k = \sigma_{K_2} \circ \pi \circ \gamma \circ \sigma_{K_1} \circ \theta \circ \pi \circ \gamma \circ\sigma_{K_0}$.

    \begin{equation*}
        \begin{aligned}
            &\sigma_{K_0}
            \begin{pmatrix}
                1110 & 1010 \\
                0110 & 0011
            \end{pmatrix} = 
            \begin{pmatrix}
                1110 & 1010 \\
                0110 & 0011
            \end{pmatrix} + 
            \begin{pmatrix}
                0000 & 1000 \\
                0110 & 0101
            \end{pmatrix} = 
            \begin{pmatrix}
                1110 & 0010 \\
                0000 & 0110
            \end{pmatrix} \\
            %
            &\gamma \begin{pmatrix}
                1110 & 0010 \\
                0000 & 0110
            \end{pmatrix} =
            \begin{pmatrix}
                0101 & 1111 \\
                0011 & 1011
            \end{pmatrix} \\
            %
            &\pi \begin{pmatrix}
                0101 & 1111 \\
                0011 & 1011
            \end{pmatrix} =
            \begin{pmatrix}
                0101 & 1111 \\
                1011 & 0011
            \end{pmatrix} \\
            %
            &\theta \begin{pmatrix}
                0101 & 1111 \\
                1011 & 0011
            \end{pmatrix} =
            \begin{pmatrix}
                0011 & 0010 \\
                0010 & 0011
            \end{pmatrix} \cdot
            \begin{pmatrix}
                0101 & 1111 \\
                0011 & 1011
            \end{pmatrix} = sin_terminar \\
            %
            &\sigma_{K_1}
        \end{aligned}
    \end{equation*}






    \newpage
    \textbf{Apartado I. \textit{Calcula $\text{E}_{\text{dni}}(0x01234567)$ usando el modo $CFB$, $r=11$, y vector de 
            inicialización $IV = 0x0001$.}}\\


\end{document}